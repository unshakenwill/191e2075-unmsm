
\documentclass{article}
\usepackage{graphicx}
\usepackage{amsmath}
\usepackage{esvect}
\usepackage{graphicx}
\begin{document}

\title{Hola, Mundo}
\author{CARLOS ALEXANDER ANDAZABAL HUAMANI}
\date{5 de Octubre de 2024}



\maketitle

\section{Comenzando}
\textbf{Hola, mundo.} Hoy estoy aprendiendo \LaTeX . \LaTeX es un excelente lenguaje para producir documentos académicos. Puede escribir matemáticas en línea, como $a^2+b^2=c^2$. También puedo darles a las ecuaciones su propia línea:
\begin{equation}
\gamma^2+\theta^2=\omega^2    
\end{equation}
Las "ecuaciones de Maxwell" son nombradas en honor a James Clark Maxwell y son las siguientes:
\begin{equations}
\begin{align}
\vec{\nabla}\cdot\vec{E} =\frac{\rho}{\epsilon_0}  && \text{Ley de Gauss} \\
\vec{\nabla}\cdot\vec{B} =0 && \text{Ley de Gauss para el magnetismo} \\    
\vec{\nabla}\cdot\vec{E} =\frac{\rho}{\epsilon_0} && \text{Ley de Faraday} \\
\vec{\nabla}\cdot\vec{B} =\mu_0(\epsilon_0\frac{\partial\vec{E}}{\partial t}+\vec{J}) && \text{Ley de Ampere}
\end{align}
\end{equations}
Las ecuaciones 2, 3, 4 y 5 son algunas de las más importantes en Física.
\maketitle

\section{¿Qué hay sobre las ecuaciones matriciales?}
\begin{align*}
\begin{pmatrix}
    a_{11} & a_{12} & \dots & a_{1n} \\
    a_{21} & a_{22} & \dots & a_{2n} \\
    \vdots & \vdots & \ddots & \vdots \\
    a_{n1} & a_{n2} & \dots & a_{nn} \\
\end{pmatrix}
\begin{bmatrix}
    v_1 \\
    v_2\\
    \vdots\\
    v_n
\end{bmatrix}
=
\begin{matrix}
    w_1 \\
    w_2\\
    \vdots\\
    w_n  
\end{matrix}  
\end{align*}
\section{Tablas y Figuras}
Crear una tabla no es muy diferente de crear una matriz
\begin{align*}
    \text{Table 1: Mi primera tabla}\\
    \begin{tabular}{|c||c|c|c|}
      \hline 
        x & 1 & 2 & 3 \\
        \hline
       \textit{ f(x)} & 4 & 8 & 12 \\
        f(x) & 4 & 8 & 12 \\
    \hline
    \end{tabular}
\end{align*}  
\includegraphics{3DGraphing04.png} \\
\centering Figura 1: Cualquier imagen
    

\end{document}